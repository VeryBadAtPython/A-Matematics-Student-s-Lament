\documentclass[11pt]{report}

\title{A Mathematics Student's Lament}

\usepackage{authblk}

\author[1]{Nicholas Arvanitellis}
\author[2]{\\Jacob Bos}
\author[3]{\\Marcel Reverter-Rambaldi}

\affil[1,2,3]{Australian National University}
\affil[3]{The University of Queensland}



\usepackage{graphicx}
\usepackage{amsmath}
\usepackage{amssymb}
\usepackage{amsfonts}
\usepackage{amsthm}

\usepackage{float}

\usepackage{multicol}
\setlength{\columnsep}{1cm}

%\usepackage{setspace}

\usepackage{xcolor}

\begin{document}
\pagenumbering{roman}
    \maketitle
    \tableofcontents
\newpage
\pagenumbering{arabic}

% ================================================================================================
% === Intro              =========================================================================
% ================================================================================================
\chapter{Introduction}




% ================================================================================================
% === Ch2 - The HSC ==============================================================================
% ================================================================================================
\chapter{The HSC}




% ================================================================================================
% === Ch3 - The SACE =============================================================================
% ================================================================================================
\chapter{The SACE}

\section{Stage 1}
\subsection{Essential Mathematics}

    According to the SACE subject outline, Stage 1 Essential Mathematics covers the following topics.
    \begin{table}[H]
        \centering
        \begin{tabular}{|l|l|}
        \hline
            1 & Calculations, time, and ratio \\ \hline
            2 & Earning and spending \\ \hline
            3 & Geometry \\ \hline
            4 & Data in context \\ \hline
            5 & Measurement \\ \hline
            6 & Investing \\ \hline
            7 & Open topic \\ \hline
        \end{tabular}
    \end{table}

\subsection{General Mathematics}

    According to the SACE subject outline, Stage 1 General Mathematics covers the following topics.
    \begin{table}[H]
        \centering
        \begin{tabular}{|l|l|}
        \hline
            1 & Investing and borrowing \\ \hline
            2 & Measurement \\ \hline
            3 & Statistical investigation \\ \hline
            4 & Applications of trigonometry \\ \hline
            5 & Linear and exponential functions and their graphs \\ \hline
            6 & Matrices and networks \\ \hline
            7 & Open topic \\ \hline
        \end{tabular}
    \end{table}

\subsection{Mathematics}

    According to the SACE subject outline, Stage 1 Mathematics covers the following topics.
    \begin{table}[H]
        \centering
        \begin{tabular}{|l|l|}
        \hline
            1 & Functions and Graphs \\ \hline
            2 & Polynomials \\ \hline
            3 & Trigonometry \\ \hline
            4 & Counting and Statistics \\ \hline
            5 & Growth and Decay \\ \hline
            6 & Introduction to Differential Calculus \\ \hline
            7 & Arithmetic and geometric series and sequences \\ \hline
            8 & Geometry \\ \hline
            9 & Vectors in the plane \\ \hline
            10 & Further Trigonometry \\ \hline
        \end{tabular}
    \end{table}


% ///////////////////////////////////////////////////////////////////////////////////////////////////
\section{Stage 2}
\subsection{Essential Mathematics}

    According to the SACE subject outline, Stage 2 Essential Mathematics covers the following topics.
    \begin{table}[H]
        \centering
        \begin{tabular}{|l|l|}
        \hline
            1 & Scales, plans, and models \\ \hline
            2 & Measurement \\ \hline
            3 & Business applications \\ \hline
            4 & Statistics \\ \hline
            5 & Investments and loans \\ \hline
            6 & Open topic \\ \hline
        \end{tabular}
    \end{table}

\subsection{General Mathematics}

    According to the SACE subject outline, Stage 2 General Mathematics covers the following topics.
    \begin{table}[H]
        \centering
        \begin{tabular}{|l|l|}
        \hline
            1 & Modelling with linear relationships \\ \hline
            2 & Modelling with matrices \\ \hline
            3 & Statistical models \\ \hline
            4 & Financial models \\ \hline
            5 & Discrete models \\ \hline
            6 & Open topic \\ \hline
        \end{tabular}
    \end{table}


\subsection{Mathematical Methods}

    According to the SACE subject outline, Stage 2 Mathematical Methods covers the following topics.
    \begin{table}[H]
        \centering
        \begin{tabular}{|l|l|}
        \hline
            1 & Further differentiation and applications \\ \hline
            2 & Discrete random variables \\ \hline
            3 & Integral calculus \\ \hline
            4 & Logarithmic functions \\ \hline
            5 & Continuous random variables \\ \hline
            6 & Sampling and confidence intervals \\ \hline
        \end{tabular}
    \end{table}

\subsection{Specialist Mathematics}

    According to the SACE subject outline, Stage 2 Mathematical Methods covers the following topics.
    \begin{table}[H]
        \centering
        \begin{tabular}{|l|l|}
        \hline
            1 & Mathematical induction \\ \hline
            2 & Complex numbers \\ \hline
            3 & Functions and sketching graphs \\ \hline
            4 & Vectors in three dimensions \\ \hline
            5 & Integration techniques and applications \\ \hline
            6 & Rates of change and differential equations \\ \hline
        \end{tabular}
    \end{table}

% ================================================================================================
% === Ch4 - The QCE =============================================================================
% ================================================================================================
\chapter{The QCE}





% ================================================================================================
% === ChX - What's Missing? ======================================================================
% ================================================================================================
\chapter{What's Missing?}




% ================================================================================================
% === ChY.1 - Stage 1 ============================================================================
% ================================================================================================
\chapter{An Alternative}
\section{Stage 1 - Year 11}

\subsection{Essential Mathematics}

\subsection{Statistical Methods}
\subsection{Calculus Methods}
\subsection{Linear Algebra Methods}
\subsection{Introduction to Analytic methods}

% ================================================================================================
% === ChY.2 - Stage 2 ============================================================================
% ================================================================================================
\section{Stage 2 - Year 12}

\subsection{Essential Mathematics}
\subsection{Life Mathematics}
\subsection{Analytical Methods}
    Analytical methods should serves to develop the analytical skills necessary for the mathematical, engineering and physical sciences and an appreciation of proof, logic and the fundamental structures of mathematics.

    Proposed topics are:
    \begin{table}[H]
        \centering
        \begin{tabular}{|l|l|}
        \hline
            1.1 & Logic and Proofs (Inc. Direct, Contrapositive, Contradiction and Induction) \\ \hline
            1.2 & Introduction to algebra and real analysis ($\varepsilon - N$ and $\varepsilon - \delta$ limit definitions, groups, permutation groups, cyclic groups. \\ \hline
            2 & Functions and graphs (with links to analysis \\ \hline
            3 & Polynomials and Complex numbers (Realization of roots of unity as a cyclic group)\\ \hline
            4 & Analytic Integration (Parts, Substitution and inverse trigonometric functions)\\ \hline
            5.1 & Analytic solutions to differential equations \\ \hline
            5.2 & Vectors and Vector Calculus in three dimensions\\ \hline
        \end{tabular}
    \end{table}



\subsection{Numerical Methods}
    Numerical methods should serves to develop the topics learned in stage with emphasis on the computational skills necessary for engineering, computer science and sciences with an appreciation of computer driven calculation.

    Proposed topics are:
    \begin{table}[H]
        \centering
        \begin{tabular}{|l|l|}
        \hline
            1.1 & Introduction to computational approaches and the julia language\\
            1.2 & Revision of common differential functions\\ \hline
            2 & Further differentiation and applications \\ \hline
            3 & Integral calculus \\ \hline
            4 & Discretization of calculus models\\ \hline
            5 & Computational linear algebra \\ \hline
            6.1 & Statistics and computation \\ \hline
            6.2 & Computational problem solving \\ \hline
        \end{tabular}
    \end{table}


\end{document}